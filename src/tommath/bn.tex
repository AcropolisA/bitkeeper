\documentclass{article}
\begin{document}

\title{LibTomMath v0.02 \\ A Free Multiple Precision Integer Library}
\author{Tom St Denis \\ tomstdenis@iahu.ca}
\maketitle
\newpage

\section{Introduction}
``LibTomMath'' is a free and open source library that provides multiple-precision integer functions required to form a basis
of a public key cryptosystem.  LibTomMath is written entire in portable ISO C source code and designed to have an application
interface much like that of MPI from Michael Fromberger.  

LibTomMath was written from scratch by Tom St Denis but designed to be  drop in replacement for the MPI package.  The 
algorithms within the library are derived from descriptions as provided in the Handbook of Applied Cryptography and Knuth's
``The Art of Computer Programming''.  The library has been extensively optimized and should provide quite comparable 
timings as compared to many free and commercial libraries.

LibTomMath was designed with the following goals in mind:
\begin{enumerate}
\item Be a drop in replacement for MPI.
\item Be much faster than MPI.
\item Be written entirely in portable C.
\end{enumerate}

All three goals have been achieved.  Particularly the speed increase goal.  For example, a 512-bit modular exponentiation is
four times faster\footnote{On an Athlon XP with GCC 3.2} with LibTomMath compared to MPI.

Being compatible with MPI means that applications that already use it can be ported fairly quickly.  Currently there are 
a few differences but there are many similarities.  In fact the average MPI based application can be ported in under 15
minutes.  

Thanks goes to Michael Fromberger for answering a couple questions and Colin Percival for having the patience and courtesy to
help debug and suggest optimizations.  They were both of great help!

\section{Building Against LibTomMath}

Building against LibTomMath is very simple because there is only one source file.  Simply add ``bn.c'' to your project and 
copy both ``bn.c'' and ``bn.h'' into your project directory.  There is no configuration nor building required before hand.

If you are porting an MPI application to LibTomMath the first step will be to remove all references to MPI and replace them 
with references to LibTomMath.  For example, substitute 

\begin{verbatim}
#include "mpi.h"
\end{verbatim}

with 

\begin{verbatim}
#include "bn.h"
\end{verbatim}

Remove ``mpi.c'' from your project and replace it with ``bn.c''.  Note that currently MPI has a few more functions than
LibTomMath has (e.g. no square-root code and a few others).  Those are planned for future releases.  In the interim work 
arounds can be sought.  Note that LibTomMath doesn't lack any functions required to build a cryptosystem.

\section{Programming with LibTomMath}

\subsection{The mp\_int Structure}
All multiple precision integers are stored in a structure called \textbf{mp\_int}.  A multiple precision integer is
essentially an array of \textbf{mp\_digit}.  mp\_digit is defined at the top of bn.h.  Its type can be changed to suit
a particular platform.  

For example, when \textbf{MP\_8BIT} is defined\footnote{When building bn.c.} a mp\_digit is a unsigned char and holds 
seven bits.  Similarly when \textbf{MP\_16BIT} is defined a mp\_digit is a unsigned short and holds 15 bits.  
By default a mp\_digit is a unsigned long and holds 28 bits.  

The choice of digit is particular to the platform at hand and what available multipliers are provided.  For 
MP\_8BIT either a $8 \times 8 \Rightarrow 16$ or $16 \times 16 \Rightarrow 16$ multiplier is optimal.  When 
MP\_16BIT is defined either a $16 \times 16 \Rightarrow 32$ or $32 \times 32 \Rightarrow 32$ multiplier is optimal.  By
default a $32 \times 32 \Rightarrow 64$ or $64 \times 64 \Rightarrow 64$ multiplier is optimal.  

This gives the library some flexibility.  For example, a i8051 has a $8 \times 8 \Rightarrow 16$ multiplier.  The 
16-bit x86 instruction set has a $16 \times 16 \Rightarrow 32$ multiplier.  In practice this library is not particularly
designed for small devices like an i8051 due to the size.  It is possible to strip out functions which are not required 
to drop the code size.  More realistically the library is well suited to 32 and 64-bit processors that have decent
integer multipliers.  The AMD Athlon XP and Intel Pentium 4 processors are examples of well suited processors.

Throughout the discussions there will be references to a \textbf{used} and \textbf{alloc} members of an integer.  The
used member refers to how many digits are actually used in the representation of the integer.  The alloc member refers
to how many digits have been allocated off the heap.  There is also the $\beta$ quantity which is equal to $2^W$ where 
$W$ is the number of bits in a digit (default is 28).  

\subsection{Basic Functionality}
Essentially all LibTomMath functions return one of three values to indicate if the function worked as desired.  A 
function will return \textbf{MP\_OKAY} if the function was successful.  A function will return \textbf{MP\_MEM} if
it ran out of memory and \textbf{MP\_VAL} if the input was invalid.  

Before an mp\_int can be used it must be initialized with 

\begin{verbatim}
int mp_init(mp_int *a);
\end{verbatim}

For example, consider the following.

\begin{verbatim}
#include "bn.h"
int main(void)
{
   mp_int num;
   if (mp_init(&num) != MP_OKAY) {
      printf("Error initializing a mp_int.\n");
   }
   return 0;
}   
\end{verbatim}

A mp\_int can be freed from memory with

\begin{verbatim}
void mp_clear(mp_int *a);
\end{verbatim}

This will zero the memory and free the allocated data.  There are a set of trivial functions to manipulate the 
value of an mp\_int.  

\begin{verbatim}
/* set to zero */
void mp_zero(mp_int *a);

/* set to a digit */
void mp_set(mp_int *a, mp_digit b);

/* set a 32-bit const */
int mp_set_int(mp_int *a, unsigned long b);

/* init to a given number of digits */
int mp_init_size(mp_int *a, int size);

/* copy, b = a */
int mp_copy(mp_int *a, mp_int *b);

/* inits and copies, a = b */
int mp_init_copy(mp_int *a, mp_int *b);
\end{verbatim}

The \textbf{mp\_zero} function will clear the contents of a mp\_int and set it to positive.  The \textbf{mp\_set} function 
will zero the integer and set the first digit to a value specified.  The \textbf{mp\_set\_int} function will zero the 
integer and set the first 32-bits to a given value.  It is important to note that using mp\_set can have unintended 
side effects when either the  MP\_8BIT or MP\_16BIT defines are enabled.  By default the library will accept the 
ranges of values MPI will (and more).

The \textbf{mp\_init\_size} function will initialize the integer and set the allocated size to a given value.  The 
allocated digits are zero'ed by default but not marked as used.  The \textbf{mp\_copy} function will copy the digits
(and sign) of the first parameter into the integer specified by the second parameter.  The \textbf{mp\_init\_copy} will
initialize the first integer specified and copy the second one into it.  Note that the order is reversed from that of
mp\_copy.  This odd ``bug'' was kept to maintain compatibility with MPI.

\subsection{Digit Manipulations}

There are a class of functions that provide simple digit manipulations such as shifting and modulo reduction of powers
of two.  

\begin{verbatim}
/* right shift by "b" digits */
void mp_rshd(mp_int *a, int b);

/* left shift by "b" digits */
int mp_lshd(mp_int *a, int b);

/* c = a / 2^b */
int mp_div_2d(mp_int *a, int b, mp_int *c);

/* b = a/2 */
int mp_div_2(mp_int *a, mp_int *b);

/* c = a * 2^b */
int mp_mul_2d(mp_int *a, int b, mp_int *c);

/* b = a*2 */
int mp_mul_2(mp_int *a, mp_int *b);

/* c = a mod 2^d */
int mp_mod_2d(mp_int *a, int b, mp_int *c);
\end{verbatim}

Both the \textbf{mp\_rshd} and \textbf{mp\_lshd} functions provide shifting by whole digits.  For example, 
mp\_rshd($x$, $n$) is the same as $x \leftarrow \lfloor x / \beta^n \rfloor$ while mp\_lshd($x$, $n$) is equivalent
to $x \leftarrow x \cdot \beta^n$.  Both functions are extremely fast as they merely copy digits within the array.  

Similarly the \textbf{mp\_div\_2d} and \textbf{mp\_mul\_2d} functions provide shifting but allow any bit count to 
be specified.  For example, mp\_div\_2d($x$, $n$, $y$) is the same as $y =\lfloor x / 2^n \rfloor$ while 
mp\_mul\_2d($x$, $n$, $y$) is the same as $y = x \cdot 2^n$.  The \textbf{mp\_div\_2} and \textbf{mp\_mul\_2} 
functions are legacy functions that merely shift right or left one bit respectively.  The \textbf{mp\_mod\_2d} function
reduces an integer mod a power of two.  For example, mp\_mod\_2d($x$, $n$, $y$) is the same as 
$y \equiv x \mbox{ (mod }2^n\mbox{)}$.

\subsection{Basic Arithmetic}

Next are the class of functions which provide basic arithmetic.

\begin{verbatim}
/* b = -a */
int mp_neg(mp_int *a, mp_int *b);

/* b = |a| */
int mp_abs(mp_int *a, mp_int *b);

/* compare a to b */
int mp_cmp(mp_int *a, mp_int *b);

/* compare |a| to |b| */
int mp_cmp_mag(mp_int *a, mp_int *b);

/* c = a + b */
int mp_add(mp_int *a, mp_int *b, mp_int *c);

/* c = a - b */
int mp_sub(mp_int *a, mp_int *b, mp_int *c);

/* c = a * b */
int mp_mul(mp_int *a, mp_int *b, mp_int *c);

/* b = a^2 */
int mp_sqr(mp_int *a, mp_int *b);

/* a/b => cb + d == a */
int mp_div(mp_int *a, mp_int *b, mp_int *c, mp_int *d);

/* c == a mod b */
#define mp_mod(a, b, c) mp_div(a, b, NULL, c)
\end{verbatim}

The \textbf{mp\_cmp} will compare two integers.  It will return \textbf{MP\_LT} if the first parameter is less than
the second, \textbf{MP\_GT} if it is greater or \textbf{MP\_EQ} if they are equal.  These constants are the same as from
MPI.

The \textbf{mp\_add}, \textbf{mp\_sub}, \textbf{mp\_mul}, \textbf{mp\_div}, \textbf{mp\_sqr} and \textbf{mp\_mod} are all
fairly straight forward to understand.  Note that in mp\_div either $c$ (the quotient) or $d$ (the remainder) can be 
passed as NULL to ignore it.  For example, if you only want the quotient $z = \lfloor x/y \rfloor$ then a call such as 
mp\_div(\&x, \&y, \&z, NULL) is acceptable.

There is a related class of ``single digit'' functions that are like the above except they use a digit as the second
operand.

\begin{verbatim}
/* compare against a single digit */
int mp_cmp_d(mp_int *a, mp_digit b);

/* c = a + b */
int mp_add_d(mp_int *a, mp_digit b, mp_int *c);

/* c = a - b */
int mp_sub_d(mp_int *a, mp_digit b, mp_int *c);

/* c = a * b */
int mp_mul_d(mp_int *a, mp_digit b, mp_int *c);

/* a/b => cb + d == a */
int mp_div_d(mp_int *a, mp_digit b, mp_int *c, mp_digit *d);

/* c = a mod b */
#define mp_mod_d(a,b,c) mp_div_d(a, b, NULL, c)
\end{verbatim}

Note that care should be taken for the value of the digit passed.  By default, any 28-bit integer is a valid digit that can
be passed into the function.  However, if MP\_8BIT or MP\_16BIT is defined only 7 or 15-bit (respectively) integers 
can be passed into it.

\subsection{Modular Arithmetic}

There are some trivial modular arithmetic functions.

\begin{verbatim}
/* d = a + b (mod c) */
int mp_addmod(mp_int *a, mp_int *b, mp_int *c, mp_int *d);

/* d = a - b (mod c) */
int mp_submod(mp_int *a, mp_int *b, mp_int *c, mp_int *d);

/* d = a * b (mod c) */
int mp_mulmod(mp_int *a, mp_int *b, mp_int *c, mp_int *d);

/* c = a * a (mod b) */
int mp_sqrmod(mp_int *a, mp_int *b, mp_int *c);

/* c = 1/a (mod b) */
int mp_invmod(mp_int *a, mp_int *b, mp_int *c);

/* c = (a, b) */
int mp_gcd(mp_int *a, mp_int *b, mp_int *c);

/* c = [a, b] or (a*b)/(a, b) */
int mp_lcm(mp_int *a, mp_int *b, mp_int *c);

/* d = a^b (mod c) */
int mp_exptmod(mp_int *a, mp_int *b, mp_int *c, mp_int *d);
\end{verbatim}

These are all fairly simple to understand.  The \textbf{mp\_invmod} is a modular multiplicative inverse.  That is it
stores in the third parameter an integer such that $ac \equiv 1 \mbox{ (mod }b\mbox{)}$ provided such integer exists.  If
there is no such integer the function returns \textbf{MP\_VAL}.

\subsection{Radix Conversions}
To read or store integers in other formats there are the following functions.

\begin{verbatim}
int mp_unsigned_bin_size(mp_int *a);
int mp_read_unsigned_bin(mp_int *a, unsigned char *b, int c);
int mp_to_unsigned_bin(mp_int *a, unsigned char *b);

int mp_signed_bin_size(mp_int *a);
int mp_read_signed_bin(mp_int *a, unsigned char *b, int c);
int mp_to_signed_bin(mp_int *a, unsigned char *b);

int mp_read_radix(mp_int *a, unsigned char *str, int radix);
int mp_toradix(mp_int *a, unsigned char *str, int radix);
int mp_radix_size(mp_int *a, int radix);
\end{verbatim}

The integers are stored in big endian format as most libraries (and MPI) expect.  The \textbf{mp\_read\_radix} and 
\textbf{mp\_toradix} functions read and write (respectively) null terminated ASCII strings in a given radix.  Valid values
for the radix are between 2 and 64 (inclusively).  

\section{Timing Analysis}
\subsection{Observed Timings}
A simple test program ``demo.c'' was developed which builds with either MPI or LibTomMath (without modification).  The
test was conducted on an AMD Athlon XP processor with 266Mhz DDR memory and the GCC 3.2 compiler\footnote{With build
options ``-O3 -fomit-frame-pointer -funroll-loops''}.    The multiplications and squarings were repeated 10,000 times 
each while the modular exponentiation (exptmod) were performed 10 times each.  The RDTSC (Read Time Stamp Counter) instruction
was used to measure the time the entire iterations took and was divided by the number of iterations to get an
average.  The following results were observed.

\begin{small}
\begin{center}
\begin{tabular}{c|c|c|c}
\hline \textbf{Operation} & \textbf{Size (bits)} & \textbf{Time with MPI (cycles)} & \textbf{Time with LibTomMath (cycles)} \\
\hline
Multiply & 128 & 1,394  & 893  \\
Multiply & 256 & 2,559  & 1,744  \\
Multiply & 512 & 7,919  & 4,484  \\
Multiply & 1024 & 28,460  & 9,326, \\
Multiply & 2048 & 109,637  & 30,140  \\
Multiply & 4096 & 467,226  & 122,290  \\
\hline 
Square & 128 & 1,288  & 1,172  \\
Square & 256 & 1,705  & 2,162  \\
Square & 512 & 5,365  & 3,723  \\
Square & 1024 & 18,836  & 9,063  \\
Square & 2048 & 72,334  & 27,489  \\
Square & 4096 & 306,252  & 110,372  \\
\hline 
Exptmod & 512 & 30,497,732  & 6,898,504  \\
Exptmod & 768 & 98,943,020  & 15,510,779  \\
Exptmod & 1024 & 221,123,749  & 27,962,904  \\
Exptmod & 2048 & 1,694,796,907  & 146,631,975  \\
Exptmod & 2560 & 3,262,360,107  & 305,530,060  \\
Exptmod & 3072 & 5,647,243,373  & 472,572,762  \\
Exptmod & 4096 & 13,345,194,048  & 984,415,240  

\end{tabular}
\end{center}
\end{small}

\subsection{Digit Size}
The first major constribution to the time savings is the fact that 28 bits are stored per digit instead of the MPI 
defualt of 16.  This means in many of the algorithms the savings can be considerable.  Consider a baseline multiplier 
with a 1024-bit input.  With MPI the input would be 64 16-bit digits whereas in LibTomMath it would be 37 28-bit digits.
A savings of $64^2 - 37^2 = 2727$ single precision multiplications.  

\subsection{Multiplication Algorithms}
For most inputs a typical baseline $O(n^2)$ multiplier is used which is similar to that of MPI.  There are two variants 
of the baseline multiplier.  The normal and the fast variants.  The normal baseline multiplier is the exact same as the
algorithm from MPI.  The fast baseline multiplier is optimized for cases where the number of input digits $N$ is less
than or equal to $2^{w}/\beta^2$.  Where $w$ is the number of bits in a \textbf{mp\_word}.  By default a mp\_word is
64-bits which means $N \le 256$ is allowed which represents numbers upto $7168$ bits.

The fast baseline multiplier is optimized by removing the carry operations from the inner loop.  This is often referred
to as the ``comba'' method since it computes the products a columns first then figures out the carries.  This has the
effect of making a very simple and paralizable inner loop.

For large inputs, typically 80 digits\footnote{By default that is 2240-bits or more.} or more the Karatsuba method is 
used.  This method has significant overhead but an asymptotic running time of $O(n^{1.584})$ which means for fairly large
inputs this method is faster.  The Karatsuba implementation is recursive which means for extremely large inputs they
will benefit from the algorithm.

MPI only implements the slower baseline multiplier where carries are dealt with in the inner loop.  As a result even at
smaller numbers (below the Karatsuba cutoff) the LibTomMath multipliers are faster.

\subsection{Squaring Algorithms}

Similar to the multiplication algorithms there are two baseline squaring algorithms.  Both have an asymptotic running
time of $O((t^2 + t)/2)$.  The normal baseline squaring is the same from MPI and the fast is a ``comba'' squaring
algorithm.  The comba method is used if the number of digits $N$ is less than $2^{w-1}/\beta^2$ which by default 
covers numbers upto $3584$ bits.  

There is also a Karatsuba squaring method which achieves a running time of $O(n^{1.584})$ after considerably large
inputs.

MPI only implements the slower baseline squaring algorithm.  As a result LibTomMath is considerably faster at squaring
than MPI is.

\subsection{Exponentiation Algorithms}

LibTomMath implements a sliding window $k$-ary left to right exponentiation algorithm.  For a given exponent size $L$ an
appropriate window size $k$ is chosen.  There are always at most $L$ modular squarings and $\lfloor L/k \rfloor$ modular
multiplications.   The $k$-ary method works by precomputing values $g(x) = b^x$ for $0 \le x < 2^k$ and a given base 
$b$.  Then the multiplications are grouped in windows of $k$ bits.  The sliding window technique has the benefit 
that it can skip multiplications if there are zero bits following or preceding a window.  Consider the exponent 
$e = 11110001_2$ if $k = 2$ then there will be a two squarings, a multiplication of $g(3)$, two squarings, a multiplication
of $g(3)$, four squarings and and a multiplication by $g(1)$.  In total there are 8 squarings and 3 multiplications.  

MPI uses a binary square-multiply method.  For the same exponent $e$ it would have had 8 squarings and 5 multiplications.  
There is a precomputation phase for the method LibTomMath uses but it generally cuts down considerably on the number
of multiplications.  Consider a 512-bit exponent.  The worst case for the LibTomMath method results in 512 squarings and 
124 multiplications.  The MPI method would have 512 squarings and 512 multiplications.  Randomly every $2k$ bits another 
multiplication is saved via the sliding-window technique on top of the savings the $k$-ary method provides.

Both LibTomMath and MPI use Barrett reduction instead of division to reduce the numbers modulo the modulus given.  
However, LibTomMath can take advantage of the fact that the multiplications required within the Barrett reduction
do not have to give full precision.  As a result the reduction step is much faster and just as accurate.  The LibTomMath code
will automatically determine at run-time (e.g. when its called) whether the faster multiplier can be used.  The
faster multipliers have also been optimized into the two variants (baseline and comba baseline).

As a result of all these changes exponentiation in LibTomMath is much faster than compared to MPI.  



\end{document}